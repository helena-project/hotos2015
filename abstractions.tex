\section{Abstractions}

The basic unit of program composition in Tock is a protection domain. A
protection domain describes a set of access rights to physical memory. 

There are be one or more event queues. Every event is associated with
a protection domain and executes within it when it begins. An event
may cross into another protection domain through a function call
(language protection) or a call gate (software protection), or
may invoke it asynchronously through a message (software protection,
hardware protection).

For example, the operating system kernel runs in one protection domain,
which has access to private kernel memory and a region of memory used to
pass messages to the kernel. The kernel memory is not accessible by other
domains, but the message memory is. The kernel reserves the IO pins and buses
that control its peripherals (e.g., the radio, the bus to another core). 

An application runs in another protection domain. It has access to its own
memory, the kernel message memory, and the subset of IO pins and buses
which the kernel does not manage/control.

The operating system can be broken into multiple protection domains. For
example, on Firestorm, any code running on the squall must be in a different
protection domain than code running on the storm core. Like a microkernel
architecture, the OS can invoke core services in separate protection domains.
The decision of whether to place drivers/etc. in the main OS protection domain
or in other domains is driven by a combination of security concerns and 
hardware capabilities.

When a developer writes an application, it consists of potentially multiple
protection domains, some in the form of libraries, some in the form of
application code, some in the form of drivers. The development environment
decides how to divide this code across the cores based on harwdware capabilites
and software operations. E.g., a filter on Bluetooth messages will likely be 
placed on the squall, while the IPv6 stack will likely be placed on the storm.
