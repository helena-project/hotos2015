\section{Protection \emph{is} a New Primitive}

% pp 33--34, ch 1, sec 4:
In Tanenbaum's \emph{Modern Operating Systems} he asserts, ``The main property
which distinguishes embedded systems from handhelds is the certainty that no
untrusted software will ever run on it''~\cite{tanenbaum}. In this paper we
argue that this is no longer the case. As a class of device, embedded systems
are evolving beyond the single-application case, and as a consequence, they
require their operating system to provide process isolation, enforced resource
arbitration, and protection.
%
Three trends, modern language design, new microcontroller hardware primitives,
and contemporary multi-MCU system design afford new opportunites to provide
protection as a primitive that was previously unavailable to embedded systems.
We are exploring these protection mechanisms in a new embedded operating system
called \name. We are designing \name specifically for embedded products that
will allow end-users to install third-party applications during deployment and
that incorporate arbitrary new devices in many configurations.

\subsection{Language Level Protection}

Embedded operating systems, like their desktop and server counterparts, have
traditionally been developed using C, or C-derived languages. These languages
lack many protection features of more modern systems languages---memory and
thread safety, a strongly enforced module system---and are inapproriate for
sandboxing untrusted linked code. This has meant that, while device drivers are
often contributed by external developers, likely receive less attention than the
core kernel code and are most likely to contain security vulnerabilities, they
run with the same privileges as core kernel modules, like the scheduler. This
problem is particularly salient in embedded systems frameworks like Arduino,
where platforms are often constructed by cobbling various device ``shields''
together and downloading or copy-pasting device driver code. While recent server
operating systems utilize hardware I/O virtualization~\cite{arrakis:osdi2014,
ix:osdi2014} to run device drivers outside the kernel, such hardware mechanisms
are unlikely to be available in microcontrollers in the foreseeable future.

\name is written in Rust: a type-safe, thread-safe, compiled, low-level systems
language. Using a language with strong safety protections enables \name to
enforce the separation between core kernel code and contributed device
drivers, which are traditionally where most bugs are found but are included in
the kernel for performance and practical reasons. Moreover, resource sharing
between drivers (e.g. of communication busses) must be done {\bf explicitly}, and
by default, the type system ensures concurrency bugs are limited to a single
device driver (i.e. a device driver can send the wrong command to a flash
storage device but cannot send a command to a radio module sharing the same SPI
bus). In particular, \name uses XX features of Rust to sandbox device drivers.

\begin{enumerate}
  \item Devices drivers are compiled with a safe subset of the Rust language
  (enforced by a compiler flag), preventing device drivers from maliciously, or
  inadvertedly subverting the type or module system.

\end{enumerate} 

\subsection{Memory Protection}

Modern ARM Cortex-M processors provide a hardware protection mechanism called
Memory Protection Unit (MPU).  An MPU allows the kernel to set access
permissions on a fixed number of memory regions which are enforced on
application code. Like Memory Management Units (MMUs) found in application
processors, MPUs trap illegal memory accesses (e.g. writing to read-only memory)
to the kernel, however unlike the MMUs, they do not provide virtual addressing,
so do not enable mechanisms like swap memory, shared libraries, etc. MPUs are
typically much more fine-grained than virtual memory. For example, in the ARM
Cortex-M series of microcontrollers, the MPU can address regions as small as 32
bytes, whereas virtual memory typically divides memory into pages of at least
4KB.

\name uses the MPU to protect kernel and driver memory from untrusted
application code as well as different applications from one another.
Applications are given dedicated region of memory for stack, heap and appliction
specific kernel buffers. When execution is yielded to an application, the MPU
restricts access to memory outside of this region. Moreover, application
specific kenel buffers (e.g. interrupt callback queues) are allocated in the
application's memory region, but protected from application tampering. This is
possible due to the fine granularity of the MPU and allows \name to eliminate
dynamic allocation in the kernel but allow more runtime flexibility in
applications.

\subsection{Multiple Processors}

Modern embedded platforms increasingly include multiple microprocessors. Often,
peripheral devices like radio modules are actually a full blown microprocessor
that to allow the application controller to offload low-level radio
functionality and reduce overall power~\cite{nrf51822,cc2540}. In other cases, a
similar microprocessor with a different energy profile to provide applications
with more control over the energy-performance tradeoff.  For example, the Nest
Protect includes a main Cortex-M4 application controller, a secondary Cortex-M0+
peripheral processor and the EM357 802.15.4 radio SoC, which has an additional
onboard Cortex-M3~\cite{nestprotect-teardown}.

\name leverages multiprocessor environments to provide protection where
language-level and memory isolation are insufficient. In a multi-application
environment, time-sensitive applications (e.g. that must process requests
within a single radio scheduling quantum) are protected from interference by,
e.g compute heavy applications. In addition, \name schedules applications on
different processors to applications handling sensitive data from side-channel
attacks by other applications.

