\subsection{Language Level Protection}

The C programming language has long been the language of choice for system-level
programming, including operating systems development. C is inherently an unsafe
programming language \cite{kint:osdi2012, undefined:apsys2012}. As such, much
effort has been made to develop operating systems in languages with stricter
semantics \cite{singularity:sigops, house:icfp2005, unikernels:2013}. These
efforts have shown that language features like type safety, memory safety, and
strict aliasing can provide support for operating system protection.
% Unfortunately, high-level languages often rely heavily on things like dynamic
% allocation and garbage collection, which make them inapproriate for an embedded
% operating system. Luckily, low-level languages are beginning to include many of
% these features~\cite{c++14,rust}, making them available to embedded OS builders.

\paragraph{Memory Safety.}
Memory safety bugs---dangling pointers, double-frees, access to unallocated
memory, and pointer arithmetic errors---plague operating systems written in
languages with weak memory semantics. Languages that guarantee memory safety
obviate such bugs by catching them at compile-time. For instance, a kernel
written in a memory safe language can enforce memory isolation of drivers at
compile time without relying on hardware support~\cite{spin:sosp}.

\paragraph{Type Safety.}
Strong, strict types enable low-level hardware to be exposed through typed
abstractions that cannot be subverted by untrusted code. Because client code
utilizing these interfaces is constrained to the particular hardware operations
exposed by the interface, careful interface design can result in fully safe
hardware access for untrusted code. As a result, careful interface design can
allow for minimum unsafe kernel code. We argue that principled methods for
designing such interfaces should be a goal of future operating systems research.

\paragraph{Strict Aliasing.}
Operating systems typically enforce, but do not guarantee, thread safety at
runtime by using atomic primitives. In contrast, languages with strict aliasing
rules, such as unique and read/write references, guarantee thread safety at
compile-time. \textit{Unique references} ensure that only a single active
execution context can access the reference, while \textit{read/write references}
ensure that multiple references to the same state are immutable. If multiple
applications require write access to the same resource, a broker with a unique
reference to the resource can mediate access. Consequently, by modeling hardware
resources as unique or read/write references, the kernel can guarantee the
preclusion of hardware race conditions.
