% TEMPLATE for Usenix papers, specifically to meet requirements of
%  USENIX '05
% originally a template for producing IEEE-format articles using LaTeX.
%   written by Matthew Ward, CS Department, Worcester Polytechnic Institute.
% adapted by David Beazley for his excellent SWIG paper in Proceedings,
%   Tcl 96
% turned into a smartass generic template by De Clarke, with thanks to
%   both the above pioneers
% use at your own risk.  Complaints to /dev/null.
% make it two column with no page numbering, default is 10 point

% Munged by Fred Douglis <douglis@research.att.com> 10/97 to separate
% the .sty file from the LaTeX source template, so that people can
% more easily include the .sty file into an existing document.  Also
% changed to more closely follow the style guidelines as represented
% by the Word sample file. 

% Note that since 2010, USENIX does not require endnotes. If you want
% foot of page notes, don't include the endnotes package in the 
% usepackage command, below.

% This version uses the latex2e styles, not the very ancient 2.09 stuff.
\documentclass[letterpaper,twocolumn,10pt]{article}
\usepackage{usenix,epsfig,xspace}
\usepackage[protrusion=true,expansion=true,kerning]{microtype}         % Prettier text
\usepackage{soul}                                                      % Provides \hl
%\usepackage[subrefformat=parens,labelformat=simple]{subcaption}        % Replaces both subfig and subfigure
\usepackage{textcomp}                                                  % Provides \textmu for upright mu's

\usepackage[usenames,dvipsnames,svgnames]{xcolor}                      % Allow the use and definition of colors
\usepackage[colorlinks=true,citecolor=violet,urlcolor=blue]{hyperref}  % Creates hyperlinks from ref/cite
\hypersetup{pdfstartview=FitH}                                         % Sets default zoom to 100% width 
\usepackage[capitalise,nameinlink,noabbrev]{cleveref}                  % Do the right thing with fig/table references


% Break URLs properly (thanks to Alex Halderman)
\def\UrlBreaks{\do-\do\.\do\@\do\\\do\!\do\_\do\|\do\;\do\>\do\]\do\)\do\,\do\?\do\'\do+\do\=\do\#}
\def\UrlBigBreaks{\do\:\do\/}


\newcommand{\name}{Tock\xspace}

% Some macros that a broadly useful:
\newcommand{\uW}{{\textmu}W\xspace}
\newcommand{\uA}{{\textmu}A\xspace}
\newcommand{\um}{{\textmu}m\xspace}
\newcommand{\us}{{\textmu}s\xspace}
\newcommand{\uF}{{\textmu}F\xspace}
\newcommand{\uJ}{{\textmu}J\xspace}
\newcommand{\iic}{I$^2$C\xspace}
\newcommand{\vdd}{V$_{\textnormal{DD}}$\xspace}

% Command for unnumbered footnote that doesn't increment footnote counter
\makeatletter
\def\blfootnote{\xdef\@thefnmark{}\@footnotetext}
\makeatother

\begin{document}

%don't want date printed
\date{}

%make title bold and 14 pt font (Latex default is non-bold, 16 pt)
\title{\Large \bf Microcontrollers Deserve Protection Too}

\author{
\begin{tabular}{ccc}
  \multicolumn{3}{c}{
    Michael P Andersen$^\dagger$,
    Tom Bauer$^\ddagger$,
    Sergio Benitez$^\ddagger$,
    Bradford Campbell$^\S$,
    David Culler$^\dagger$,
  } \\
  \multicolumn{3}{c}{
    Prabal Dutta$^\S$,
    Philip Levis$^\ddagger$,
    Amit Levy$^\ddagger$,
    and
    Pat Pannuto$^\S$
    \vspace{0.3cm}
  } \\
  %
  \small{$^\dagger$Computer Science \& Engineering} &
  \small{$^\ddagger$Computer Science \& Engineering} &
  \small{$^\S$Computer Science \& Engineering} \\
  %
  \small{University of California, Berkeley} &
  \small{Stanford University} &
  \small{University of Michigan} \\
  %
  \small{Berkeley, CA 94720} &
  \small{Stanford, CA 94305} &
  \small{Ann Arbor, MI 48109} \\
  %
  \small{\{m.andersen,culler\}@berkeley.edu} &
  \small{\{tbauer01,sbenitez,pal,levya\}@stanford.edu} &
  \small{\{bradjc,prabal,ppannuto\}@umich.edu} \\
\end{tabular}
}

\maketitle

\blfootnote{Authors listed in alphabetical order.}

% Use the following at camera-ready time to suppress page numbers.
% Comment it out when you first submit the paper for review.
%\thispagestyle{empty}

\begin{abstract}


Microcontroller operating systems and frameworks typically assume that a single,
monolithic application will run on an embedded system.
Traditionally,
optimizing for power, squeezing applications into minimal code and memory
allocations, and lacking hardware support
% In fact, historically, power draw, memory and code sizes, and lack of hardware support
have
constrained microcontroller applications to be single-function.
Newer hardware, however, has changed this paradigm.
The microcontroller is growing up and can now support
a secure, trusted kernel and
multiple, isolated, concurrent, and dynamically-loaded applications,
all while operating on the power budgets that originally made this device
class feasible.
While the hardware support is present, the software ecosystem to capitalize on these
advances is lagging behind. To remedy this, we propose Tock, a new embedded operating system design
that builds on established operating system
principles while adapting to the embedded system environment. Tock exploits
memory protection units, advancements in modern systems programming languages,
and the event driven nature of embedded applications to allow a core kernel,
device specific drivers, and untrusted applications to coexist on a single
microcontroller. This new operating system will allow embedded devices to
mature beyond program-once, deploy-once systems and into re-usable, ubiquitous,
and reliable computing platforms.




% This is no
% longer the case with modern hardware. Modern microcontrollers, like the Atmel
% SAM4L (an ARM Cortex-M4), provides over six times more SRAM and over five times
% larger flash than the TI~MSP430 that powered the TelosB motes while maintaining
% similar power draws (90~{\uA}/MHz active and 3~\uA at sleep).
% Simultaneously embedded products are becoming a development platforms and an
% application ecosystems for products like the Pebble watch. Additionally, a set
% of modular embedded devices is emerging. Devices like SimBand, Wzzard and
% ThinkingThings have a core part to which additional modules can be attached to
% extend the functionality. These modules can be developed by the third parties
% and have own microcontroller with runtime environment.
% The operating systems community should leverage advancements in hardware,
% programming languages as well as our experience from the Web, and other
% application rich ecosystems, to build the next generation of embedded operating
% systems.


\end{abstract}

\section{Introduction}

% This may be dipping a little too far back into history and waxing a little
% too poetic as a consequence
In 1882, Charles Babbage designed the first automated computer, the Difference
Engine. Fifteen years later, he evolved the design to create a general-purpose
machine, the Analytical Engine. Since the birth of computing, new
computational platforms have followed this trend, beginning as specialized
devices and evolving to general purpose platforms.
Most recently, the mobile phone evolved from a single-purpose voice
communication platform to add text communication, to run small Java applets,
and eventually to the full-fledged computing devices they are today, with
voice communication acting as simply another application on the device.

Today, we see the emergence of ``intelligent'' things at varying steps along
this progression: Smart watches with a clock application and a limited set of
other applets, smart light bulbs that support single functions---on and off,
brighter and dimmer---, smarter light bulbs that pulse with the ambient music.
%
Traditionally, these embedded devices have been single-purpose
single-application devices. As a result, embedded ``operating systems'' have
generated monolithic program images, running a single application and
providing unfettered access to underlying hardware.
%
It is time for embedded operating systems to evolve from their
application-specific nature to a general-purpose operating system,
%We argue that the time has come for a true embedded operating system,
one that enables and encourages multiple applications, provides both hardware
abstractions and low-level hardware access, and offers critical services for
the applications of the Internet of Things.
%Today's fitness tracker can easily double as a medication reminder, if only
%there was a means to install and run another application.

% Hue: STM32F + CC2530 (zigbee radio)
% Pebble: Also an ST M3
% Jawbone: MSP430F5528
% Fitbit: STM 32L151C6 (another M3) + nRF8001

% THOUGHT:
%
% We make an indirect argument here by designing and OS for ARM Cortex-M.
% Namely that we are leaving other archiectures (pic, msp, avr) behind. There
% is precedent for this, x86 won as a general purpose archiecture, a key step
% towards moving forward in the embedded space is picking an archiecturural
% winner, committing to it, and moving forward. We are doing this implicitly,
% perhaps we should consider adding some text and making it an explicit choice
% somewhere.

We introduce \name, a next-generation operating system for embedded systems.
Modern microcontrollers are both feature rich, with 32-bit processors and
diverse peripherals, and feature poor, with limited memory, energy-hungry
storage and communication, and restricted hardware support (MPUs instead of
MMUs, if at all).
\name targets the Cortex-M series of microprocessors.  \name leverages as many
or as few hardware primitives are available and provides isolation and
protection between applications and the kernel using three mechanisms at
different levels:

\begin{enumerate}
  \item \emph{Language level protection} \name is written in Rust, a type-safe,
  compiled, low-level language. \name enforces isolation between it's
  internal components (e.g. core device drivers) using language-level
  protection~\endnote{Yeah, I know, this is really vague right now}.
  
  \item \emph{Software-managed memory protection} Many ARM Cortex-M
  processors provide a hardware protection mechanism called Memory Protection
  Unit (MPU).  An MPU allows the kernel to set access permissions on a fixed
  number of memory regions which are enforced on application code. Unlike the
  Memory Management Units (MMUs) in application processors, MPUs do not provide
  virtual addressing, however, like MMUs illegal accesses to memory regions
  (e.g. writing to read-only memory) are caught and result in a fault to the
  kernel. \name uses the MPU to protect kernel and driver memory from untrusted
  application code and different applications from one another, while
  providing applications direct access to hardware registers, such as the
  GPIO.

  \item \emph{Physically isolated cores} Modern embedded platforms often
  involve multiple microprocessors in practice. For example, both the CC2540
  and the NRF51822 Bluetooth modules are, in fact, full blown microprocessors,
  but many products include them in addition to another
  microprocessor~\endnote{Well, at least Coin does that}. \name leverage
  multiprocessor environments to enforce another layer of isolation between
  applications and the kernel, by running different application components on
  different cores depending on their needs to access specific hardware, as
  well as power and performance constraints.

\end{enumerate}


\section{Operating System Considerations}
\label{os-considerations}
%\subsection{Evolved Embedded Applications}

In a first evolutionary cycle of the embedded devices applications were written
as a essential part of a runtime environment on the device. During secondary
cycle operating systems like TinyOS, Contiki, FreeRTOS emerged separating
application and OS spaces. The development became easier  and applications
got complex but the resulting code was still compiled to a monolithic image and
uploaded to the device.

Now the third evolutionary cycle is emerging: multiple and independently
developed application has to co-exist on the same device (like in Pebble watch);
and embedded devices are becoming modular. The later is evident with
multi-billion companies like Samsung, Telefonica and GE innovating with SimBand, Wzzard, ThinkingThings, Spotter UNIQ.
%~\endnote{http://www.samsung.com/us/globalinnovation/innovation_areas/}, Wzzard
%~\endnote{http://bb-smartsensing.com/wzzard-sensing-platform/}, ThinkingThings
%~\endnote{http://www.thinkingthings.telefonica.com/}, Spotter UNIQ
%~\endnote{https://www.quirky.com/shop/982-spotter-uniq-customizable-multipurpose-sensor}
The aim is to create a core hardware platform where adding sensory
peripherals is done by consumer in a Lego-like fashion while developer compiles
application with provided SDK.

These trends completely change the way applications for the embedded devices
have been developed last five decades. Developer have very little control on the
application execution model, its coexistence with other application and what
peripherals are connected to the embedded device. Instead application developer
expect that underlying operating system will perform fair resource allocation,
protect the application from malicious or malfunctioning applications and
peripherals. 

Given the paradigm shift in embedded devices we design new operating system
\name that supports a combinations of goals:


\begin{itemize}
  \item Event-driven execution model
  \item Loadable applications, services and drivers
  \item Robust, Reliable and Safe %isolation
  \item Energy Efficient 
\end{itemize}



\subsection{Application execution model}
\begin{figure}
 \centering
\includegraphics[width=1\columnwidth]{img/appcycle.png}
\caption{Runnable life cycle.}
 \label{fig:appcycle}
\end{figure}


\subsection{Loadable applications, services and drivers}

\name provides a system-call interface. This
allows developers to write applications in any language that can support the
ABI. For example, the \name kernel is written in Rust~\cite{rust}, applications
can trivially written in C, and we are working on a port of Lua~\cite{lua}.

\subsections{Robust, Reliable and Safe}
%tobe rewriten
 \name follows previous operating systems in separating
drivers for core peripherals (SPI, USART, GPIO, etc) and device drivers (radio,
flash, etc) into separate layers. \name differs from previous operating systems
in enforcing safety policies on device drivers through the Rust type system.
\name prevents drivers from subverting Rust's memory safety by restricting
device drivers to a safe subset of the Rust language.~\endnote{Rust allows code
to circumvent the type system using the \tt{unsafe} keyword. \name uses a
compiler flag that disallows this keyword when compiling device drivers.} \name
also ensures, at compile time, that at most one driver has access to a specific
hardware resource---multiplexing must be done explicitly in the core peripheral
driver or through an intermediate interface. Finally, \name ensures device
drivers cannot corrupt kernel memory through careful choice of interfaces. \name
ensures that \name does not protect the kernel from denial of service attacks by
drivers.

{\bf Reliability}. Unlike most desktop and server applications, embedded
applications must continue to run without end-user intervention. There is no
console to indicate to the user that an application has crashed. Even if a crash
could be communicated to the user, there is little action she could take. While
a Blue-Screen-Of-Death is annoying on a desktop or server, it is unacceptable in
embedded systems. Unlike other embedded operating systems, in \name,
applications cannot corrupt the kernel or other applications. Moreover, certain
parts of the kernel (e.g. contributed device drivers) are isolated using
a strong language type system.

\subsection{Energy Efficient}


\subsection{Scheduling}



\section{Protection \emph{is} a New Primitive}
\label{protection}

% pp 33--34, ch 1, sec 4:

%
This section examines recent advances and trends in embedded hardware and
programming languages that enable novel protection primitives in constrained
systems.

While traditional operating systems handle independently developed applications,
these OSes cannot simply be applied to the embedded domain. They assume virtual memory
for protection and isolation and are not designed to take advantage of memory protection subsystems
present in embedded hardware. They trust that drivers are written correctly,
granting them full access to kernel structures and memory, and fail when driver bugs
are present. Lastly, they are designed for cooperative cores with shared memory,
rather than physically isolated microcontrollers with private memories.


\subsection{Hardware Advances and Trends}

New embedded system platforms will be built on the Cortex-M series of microcontrollers.
While ARM's A-series is substantially more capable and is used in wall-powered
devices such as the Raspberry Pi~\cite{rpi} and BeagleBone Black~\cite{bbb},
its significant power draw (over 1~W in active mode) make it infeasible for
low-power, battery operated devices.
In contrast, Cortex-M series microcontrollers have power draws conducive to
low-power operation while adding hardware features that enable protection.


% Modern ARM devices come in three flavors: the powerful A-series
% microprocessors, the real-time R-series, and the efficient M-series
% microcontrollers.
% \hl{XXX: Why M over A}

{\bf Memory Protection Units.}
ARM's Cortex-M is a microcontroller design---a System-on-Chip with tightly
integrated memory---and does not (nor will it likely ever) integrate enough
memory to merit a Memory Management Unit (MMU). Instead, the Cortex-M series
includes a Memory Protection Unit (MPU), a lightweight, efficient subsystem
that provides the memory protection features similar to those found in an MMU
(e.g. it will trap illegal memory accesses such as writing to read-only memory)
but without address translation and at a much finer granularity.
With OS support an MPU can enable isolated applications, shared libraries, and
even possible relocation of running code.
% Pat: I'm 90% sure of how to do relocation with PIC, but I need to think about it more
MPUs are much more fine-grained than typical virtual memory. MPUs are able to
address regions as small a 32~bytes, whereas MMUs typically use at least 4~KB
pages. Moreover, MPUs support regions of any size that is a power of two larger
than 32 bytes (27 sizes in practice)~\cite{cortexm4-ug:ch4.5}, whereas MMUs
support a limited number of page sizes.\footnote{The Itanium architecture
supports eight page sizes, while x86 and ARMv7 both only support three.}
MPUs do not, however, perform any translation, so mechanisms such as swap and
dynamic page allocation may not be feasible.

{\bf Multi-Processor Platforms.}
Modern embedded platforms increasingly include multiple microcontrollers. For
example, the Nest Protect~\cite{nest-protect} includes a main Cortex-M4 application controller, a
secondary Cortex-M0+ peripheral processor, and an EM357 802.15.4 radio SoC with
an onboard Cortex-M3~\cite{nestprotect-teardown}. Adding additional
microcontrollers can allow offloading timing-critical communication
functionality or running certain computations on a microcontroller with a
different energy profile. Multiple processors provide an opportunity for the
strictest isolation, stochastic and deterministic real-time schedules, and
computation offloading.
Existing operating systems have not needed to provide abstractions for multiple
microcontrollers and how to create these abstractions is an open question.
% It is less clear, however, how to abstract system-specific hardware for general
% applications and is an open question in the design of a new operating system.


% ARM TrustZone is also functionally like having a separate core, or at least
% that's the abstraction it tries to provide. Only available on Cortex-A's
% though

% Could also be interesting to talk about crypto co-processors here



%% OLD TEXT: impl-heavy
% \name leverages multiprocessor environments to provide protection where
% language-level and memory isolation are insufficient. In a multi-application
% environment, time-sensitive applications (e.g. that must process requests
% within a single radio scheduling quantum) are protected from interference by,
% e.g compute heavy applications. In addition, \name schedules applications on
% different processors to applications handling sensitive data from side-channel
% attacks by other applications.

%Advances:
%  MPUs
%    -- capabilities
%    -- MPU vs MMU (seed of both?)
%
%Trends:
%  M vs A
%    -- Why deeply embedded (M) still relevant
%    -- Maybe Pebble vs other watches example?
%
%  Multi-Core (really Multi-MCU?)
%    -- solid protection / boundary
%    -- compare with TrustZone?
%    -- App / HW specific; need to expose but can't feature / focus / rely on


\subsection{Language Level Protection}

The C programming language has long been the language of choice for system-level
programming, including operating systems development. C is inherently an unsafe
programming language \cite{kint:osdi2012, undefined:apsys2012}. As such, much
effort has been made to develop operating systems in high-level languages with
stricter semantics \cite{singularity:sigops, house:icfp2005, unikernels:2013}.
Type safety, memory safety, and strict aliasing each provide support for
operating system protection at the language level.

\textbf{Strong Type Safety.}
Strong, strict types allow for abstractions of low-level hardware machanisms
that cannot be subverted by safe code. Because the type system cannot be
subverted, hardware mechanisms modeled as type values exposing typed interfaces
guarantee that users of the interface are utilizing the underlying hardware in a
safe manner. Of course, the underlying interface implementation continues to be
responsible for implementing the desired logic correctly. As such, it is
important to design interfaces that minimize the ability for logic errors to be
introduced via their implementation. We argue that through careful interface
design, unsafe code, or code that interfaces with the hardware directly, can be
minimized, reducing the amount of code that must be audited to ensure
correctness. Identifying the minimum amount of unsafe code to expose a safe
interface is a core challenge, but successfully doing so results in strong,
cost-free protection.

\textbf{Memory Safety.}
Issues with memory safety have long plagued operating systems written in
languages with weak memory semantics. Dangling pointers, use-after-free and
double-free errors, access to unallocated memory, and pointer arithmetic errors
are a few of the issues kernel developers encounter when writing in such a
language. A language that guarantees memory safety ensures that some or all of
these types of errors cannot occur once a program has compiled.

A memory safe application can only access memory that has been allocated to it.
This means that a kernel with drivers written in a memory safe language are
memory isolated without the need for hardware support: It is impossible for a
memory-safe driver to sully the integrity of a kernel written in the same
language.

% A high-level language usually guarantees memory safety through automatic garbage
% collection and bounds checking; the language allocates and frees all memory for
% the user and checks all pointer arithmetic. Because the garbage collector tracks
% active references, automatic garbage collection imposes a runtime performance
% penalty that is difficult to determine deterministically. In a constrained
% system, where applications may require tight bounds on their execution,
% deterministic performance is critical. As such, automatic garbage collection is
% unlikely to be a good fit for embedded systems.

Of course, to program hardware, the kernel must be able to access memory
directly. As with type safe interfaces, the ideal is to provide a memory safe
interface to carefully audited memory unsafe operations. This guarantees the
memory safety of software utilizing the memory safe interfaces, such as drivers,
while enabling the kernel's low-level implementation.

\textbf{Strict Aliasing.}
Strict aliasing rules, such as unique references and read/write references, are
used to enforce thread safety in modern programming languages. We seek to
exploit these semantics to provide thread-safe hardware access by modeling
hardware resources as references. In this way, two or more applications can run
concurrently without fear of race conditions when accessing hardware.

% This means that two or more concurrently running applications may
% attempt to access the same hardware, resulting in access patterns unanticipated
% by the hardware. We seek to exploit strict aliasing semantics, such as unique
% and read/write references, by modeling hardware resources as references,
% guaranteeing thread-safe hardware access.

\textit{Unique references} guarantee that only a single active execution context
can access the reference. By modeling hardware as a unique reference, a
guarantee of thread safety can be made to applications holding the reference.
\textit{Read/write references} guarantee that two active contexts cannot mutate
the same state. If two or more applications require read access to the same
resource, read/write references can be used to make the same guarantee. If two
or more application require write access to the same resource, a broker with a
unique reference to the resource can mediate access to said resource.

By composing type safety, memory safety, and strict aliasing, a next-generation
operating system can expose safe interfaces to applications and device drivers
with strong protection guarantees at little performance cost.



\section{\name: a Secure Embedded OS}

\name is a new embedded operating system
specifically designed for platforms running multiple, potentially untrusted
applications concurrently and third-party, contributed device drivers that are
assumed to be buggy. \name uses three protection mechanisms to protect three
different components in the operating system.
First, the kernel is written in Rust, a new systems language that provides
compile-time memory safety and ownership, and uses the language to protect the
core of the kernel (e.g. the scheduler and hardware abstraction layer) from
contributed device drivers.
Second, \name uses new Memory Protection Units (MPUs) to isolate third-party
applications from each other and the kernel.
Finally, where available, \name uses multiple microcontrollers to protect
applications with timing concerns against starvation or leaking side-channel
information.
The remainder of this section gives an overview of the \name architecture,
and aims to provide some intuition for importance and usage of an MPU.

\subsection{Architecture}

In traditional embedded operating systems, a single application is
compiled along with a library operating system to a single executable that is
loaded on to the target device. The application, device drivers, scheduler, and
other OS services run with the same CPU privileges, share memory and have equal access
to the same hardware resources. \name takes a radically different approach. In
\name, there is a standalone, portable kernel that is composed, at compile-time,
with approriate device drivers for the platform and enables dynamic loading of
third-party applications at runtime.

At a high level, \name has three layers of computational units. At the lowest
layer, the core of the kernel---the scheduler, task manager, hardware
abstraction layer, etc---has completely control of the hardware, including
memory-unsafe and thread-unsafe operations such as writing to arbitrary memory
and turning interrupts on and off. A device driver runs with the same
\emph{hardware} privileges as the core kernel but inside a compile time,
\emph{language-level} sandbox. The language sandbox enforces that device drivers
can only use the kernel provided hardware abstractions to access hardware, that
dynamic memory allocation is limited to load time and stack allocation, and that
drivers cannot interfere with each other by accidentally sharing underlying
hardware resources. For example, the Rust language has a feature called
``ownership`` (which originally appeared in the Cyclone language~\cite{cyclone})
that ensures, at compile time, there is only one active reference to a
value not marked copyable. In \name, we use this feature to give device drivers
``ownership'' over hardware resources and ensure those resources are not a
vector for race conditions between drivers. Moreover, \name relies on Rust's
strong module boundries and type system to restrict device drivers from
accessing unsafe internals of the hardware abstraction layer.

The third layer, where applications run, is isolated using hardware protection
mechanisms. Applications, which do not have to be written in Rust, run in an
unprivileged CPU mode that restricts access to privileged CPU and hardware
registers, such as the control and program status CPU registers and the memory
protection unit hardware registers. Applications can access hardware resources
in two ways. First, through a system call interface exposed by the kernel and
device drivers.  Second, the kernel can give an applications direct read and/or
write access to particular hardware registers.

In cases where the platform has multiple microporcessors, \name can use
additional CPUs to isolate applications at yet another layer. We envision that
in multi-application embedded platforms, most applications will not be very
sensitive to reasonable scheduling decisions, however some applications will
undoubtedbly have timing constraints. \name will leverage multiple
microprocessors by scheduling applications on the microprocessor with the most
relevant hardware---for example, a communication bound application will be
scheduled on a Bluetooth SoC. Moreover, some microprocessors on a platform will
run instances of \name with a real-time scheduler. Finally, we plan to leverage
physically separate microprocessors to protect applications dealing with
particularly sensitive data (such as encryption keys, or end-user medical
information) from side-channel attacks by running them on a separate
processor~\cite{trustzone}.

% \subsection{The Cortex-M MPU}
% 
% The Cortex-M MPU allows the kernel to defined eight memory regions. Each memory
% region may be sized between 32 bytes and 4GB, in 32 byte increments and has
% protection bits for read, write and execute. Regions may overlap, in which case
% the memory region with the highest number ``wins''. In addition, regions of at
% least 256 bytes can be divided into eight equal sized subregions, which can
% either be turned on---in which case they inheret the parent region's protection
% bits---or turned off---in which case the parent's protection bits do not apply.
% As a result, the operating system can control access to up to 64 concurrently
% active regions. Finally, memory regions and protection bits can be changed
% during exection, for example while context switching between applications.

\subsection{OS Design Challenges}

To design an operating system for today's embedded hardware running multiple
untrusted applications concurrently we must rethink some of the core mechanisms
used in modern operatingy systems for other environments (specifically, the
desktop, mobile and server). For example, the main mechanism for isolating
applications from each other and the kernel in modern desktop operating systems
is virtual memory or segmentation. These mechanisms are inapproriate in an
embedded system in several ways, not least of which that hardware support for
such mechanisms is simply not available.

On the other hand, new microporcessors do support memory protection, which
unlike virtual memory, exposes a flat address space to applications and the
kernel, but allows the kernel to set access control rules on memory regions as
small as 32 bytes. \name explores one point of the design space of how to best
layout kernel and application memory, design a system call interface and mediate
access to hardware using an MPU.

Figure~\ref{fig:memory-layout} shows the four types of memory regions in \name
as well as the access rules when an application is running for those regions and
subregions.

\begin{enumerate}
  \item The {\bf code} region contains kernel, device driver and application
    code. While an application is executing code, is has read and instruction
    fetch access only to it's own code segment. It cannot read code for other
    applications or the kernel since a compiler might inline sensitive constants
    which could then appear in the code section. An application cannot write to
    it's own code section since, in most cases, the code is backed by internal
    flash, which has limited write cycles.

  \item {\bf Hardware memory registers} are typically located in a single,
    continuous region of memory.~\endnote{the location of hardware memory
      registers and flash is determined by the chip manufacturer, however, chips
      chips typically place flash at the bottom of memory, followed by RAM, and
    peripheral memory registers towards the top of memory.} \name may provide
    read-only, or read-write access over small ranges of memory registers to
    applications. For example, our development platform uses the Atmel SAM4L
    Cortex-M4 which provides a true random number generator peripheral. The
    relevant registers for reading a new random value are contained in an 20
    byte range.  \name exposes random numbers directly to applications by allow
    read-only access to a 32 byte containing the relevant registers (the tailing
    12 bytes are innocuous).

  \item The {\bf Kernel's stack and static data} are located at the top of RAM.
    They include all kernel and device driver buffers and any local variables so
    applications may not access the kernel stack in any way. The kernel does not
    allocate memory dynamically except on the stack and inside application
    memory regions, as discussed below. As a result, the kernel stack is the
    only region dedicated to kernel memory.

  \item \name allocates {\bf application memory} in the remaining space below
    the kernel stack. Currently, \name allocates a continuous, fixed size region
    for each application (our development platform has 64KB of RAM and we
    currently use 2KB memory regions for applications, but this will likely vary
    based on the requirements of each target platform). Application memory and
    the kernel stack grow towards each other to provide the most flexibility in
    number of applications vs. kernel stack size. If the kernel stack grows too
    large, \name always prefers to terminate the application with the largest
    memory region. An alternative memory allocation technique would be to
    allocate application memory as requested from a shared heap (which is
    possible due to the high granularity of the memory protection unit). We have
    not yet explored the tradeoffs between these two strategies, but at a high
    level, but fixed sized application memory regions makes it easier for the
    kernel to reclaim stack space under memory pressure while a shared heap
    provides heterogenously sized application with more flexibility.

\end{enumerate}

An application's memory region is not entirely it's own. Most of it's memory
used for it's stack and static variables (at the bottom), and heap. However, the
kernel may also ``borrow'' memory from the top of an application's memory region
for application-specific kernel data structures. This region, which may change
in size, is \emph{inside} the application's memory space, but is marked as read
and write protected when the application is running. Therefore, the kernel, or device
drivers, can allocate memory dynamically in response to application without
sacrificing the reliability of never dynamically allocating in the kernel itself
and while mainting integrety and confidentiality of shared kernel data
structures. For example, when an application registers to a timer callback, the
timer device driver adds the request to it's list of pending timers by
allocating the new list node to in the application's memory. Since the link includes
forward and backward list pointers, it's imperative that despite being in the
application's memory, this link's integrity be maintained. Moreover, the values
of the links might leak information about other applications.


\section{Related Work}

Embedded systems have historically had highly fragmented application domains,
each with a vertically integrated set of software and hardware
% that
% applications can customize as needed. However, until now, a common
% assumption in almost all embedded systems is
that a single group of developers
is responsible for.
%all of the code running on a device.
This complete oversight has typically meant
% They might borrow
% or use libraries or operating systems from elsewhere, but that code is
% completely under their control to modify and inspect as needed. Making
% this assumption means
that embedded operating systems have
minimal or no system security to protect themselves from malicious code
or applications.

One extreme example of this approach is TinyOS~\cite{tinyos}, which makes no
distinction between application and operating system code. Application
code specifies which abstractions and services it needs from the
operating system, and the compilation toolchain uses this information
to build an application-specific version of the OS that is compiled
and directly linked with application code.
Even with the threading library, TOSTheads~\cite{tosthreads}, and
dynamically linked applications, the
% Later
% versions of TinyOS included a threading library with which one could
% dynamically load and link applications~\cite{tosthreads}. These
% threads, however, operate under the assumption that the
application
developer and the system maintainer are assumed to be the same person
and applications
% : there are no
% security checks and since there is no memory protection applications
have unlimited access to the entire system.

The Contiki~\cite{contiki} operating system is another microcontroller-based
operating system
with extensive networking support.
% . It has led TCP/IP integration with embedded systems
% through libraries such as the $\mu$IP stack and a fully
% certified IPv6 stack contributed by Cisco in 2008~\cite{contiki}.
Security in Contiki is limited to
link-layer encryption and integrity and the OS does not support
dynamically loaded applications.
%% again this is not true contiki supports dynically loaded code.

FreeRTOS~\cite{freertos} is an open-source commercial operating system intended to
support a wide range of applications on many different processors. It
is used in applications from the Nest Protect to smart meters to
thermal monitoring systems. FreeRTOS supports for network security
through TLS~\cite{tls} and DTLS~\cite{dtls}, but operates under the
assumption that all code is trusted. For example, any task in the system
can spawn a task that runs in privileged mode and arbitrarily modify
% http://www.freertos.org/a00125.html
% http://sourceforge.net/p/freertos/discussion/382005/thread/7b135e95
the system~\cite{rtos-tasks,rtos-sec}.

Other embedded operating systems have explored designs with a core kernel and
separate applications. For example, SOS~\cite{han05sos} provides dynamically
loadable applications and drivers on top of a core kernel. However, given
the hardware constraints when SOS was developed, it provided no memory protection
and assumed trust between applications.
%\section{Hardware Considerations}

\begin{enumerate}
\item \bf{Memory Protection Units (MPUs)}

\item \bf{Many More Peripherals}

\item \bf{Multi-core Platforms}
\end{enumerate}

%\section{Challenges}
\begin{enumerate}
  \item The compiler and OS should synthesize a multi-core execution plan
  subject to developer configuration/annotation in the code and, potentially,
  real time constraints.

  \item What is the execution model?

  \item The kernel must provide a way of inspecting the state of the system
  during debugging. For example, in TinyOS, timer callbacks are determined
  statically, but this would not be the case in a syscall based system with
  dynamic applications. The kernel should therefore provide a way to query,
  e.g., which timer's are currently outstanding.

\end{enumerate}

%\section{Abstractions}

The basic unit of program composition in Tock is a protection domain. A
protection domain describes a set of access rights to physical memory. 

There are be one or more event queues. Every event is associated with
a protection domain and executes within it when it begins. An event
may cross into another protection domain through a function call
(language protection) or a call gate (software protection), or
may invoke it asynchronously through a message (software protection,
hardware protection).

For example, the operating system kernel runs in one protection domain,
which has access to private kernel memory and a region of memory used to
pass messages to the kernel. The kernel memory is not accessible by other
domains, but the message memory is. The kernel reserves the IO pins and buses
that control its peripherals (e.g., the radio, the bus to another core). 

An application runs in another protection domain. It has access to its own
memory, the kernel message memory, and the subset of IO pins and buses
which the kernel does not manage/control.

The operating system can be broken into multiple protection domains. For
example, on Firestorm, any code running on the squall must be in a different
protection domain than code running on the storm core. Like a microkernel
architecture, the OS can invoke core services in separate protection domains.
The decision of whether to place drivers/etc. in the main OS protection domain
or in other domains is driven by a combination of security concerns and 
hardware capabilities.

When a developer writes an application, it consists of potentially multiple
protection domains, some in the form of libraries, some in the form of
application code, some in the form of drivers. The development environment
decides how to divide this code across the cores based on harwdware capabilites
and software operations. E.g., a filter on Bluetooth messages will likely be 
placed on the squall, while the IPv6 stack will likely be placed on the storm.


{\footnotesize \bibliographystyle{acm}
\bibliography{bibliography}}

\end{document}


